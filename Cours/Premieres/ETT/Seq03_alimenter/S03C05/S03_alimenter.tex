\documentclass[10pt,fleqn]{article} % Default font size and left-justified equations
\usepackage{import}
\usepackage[%
    pdftitle={Energie et puissance d'un smartphone},
    pdfauthor={Geoffrey Vaquette}]{hyperref}
\subimport{../../../../style/}{preambule}
%\fichetrue
\fichefalse

\proftrue
%\proffalse

\tdtrue
%\tdfalse

\courstrue
%\coursfalse

\subimport{../../../../style/}{new_style}
\subimport{../../../../style/}{macros_SII}
\subimport{../../../../style/}{preambule_trou}

\usepackage{siunitx}
% -------------------------------------
% Déclaration des titres
% -------------------------------------

\def\discipline{Enseignement \\Technologique \\ Transversal}
\def\xxtete{Enseignement Technologique Transversal}

\def\classe{1 STI2D}
\def\xxnumpartie{Séquence 2}
\def\xxpartie{Energie électrique et puissance d'un smartphone}

\def\xxnumchapitre{Séance 3}
\def\xxchapitre{\hspace{.12cm} Énergies, Puissances et rendement}

\def\xxposongletx{2}
\def\xxposonglettext{1.45}
\def\xxposonglety{23}
\def\xxonglet{Seq. 2 -- TD 3}

\def\xxactivite{TD}
\def\xxauteur{\textsl{Geoffrey Vaquette}}

\def\xxcompetences{%
\textsl{%
\textbf{Savoirs et compétences :}
\begin{itemize}[label=\ding{112},font=\color{ocre}]
\item CO2.1	Identifier les flux et la forme de l'énergie, caractériser ses transformations et/ou modulations et estimer l'efficacité globale d'un système.
\end{itemize}
%
}}

\def\xxfigures{
\begin{center}

\end{center}
}%figues de la page de garde
\def\xxpied{%
Energies, Puissances et Rendement \xxactivite%
}

%---------------------------------------------------------------------------

\renewcommand{\RemplirTrou}{true}
\title{Le bloc «alimenter» du robot mbot}
\date{}
\begin{document}
\maketitle
\chapterimage{}
%\input{Cours/style/new_pagegarde}
\begin{obj}
  L'objectif de cette activité est de caractériser l'alimentation du robot \textbf{mbot}.
\end{obj}
\section{Introduction}
Vous avez à votre disposition le manuel d'utilisation du robot \textit{mbot}. Ce document est le support pour les questions suivantes.
\begin{question}
  En vous aidant de la documentation, identifiez les deux options d'alimentation du robot \textit{mbot}.
\end{question}
\vspace{\stretch{1}}
\section{Alimenter le robot à l'aide de piles}
\begin{question}
  A partir du support de piles, identifiez si ces piles sont branchées en série ou en parallèle.
\end{question}
    \vspace{\stretch{1}}
\begin{question}
  Représentez, sous la forme d'un schéma électrique le branchement des piles :
\end{question}
    \vspace{\stretch{3}}

\begin{question}
  Chaque emplacement peut contenir une pile AA. Quel est la tension d'une telle pile ? A l'aide des questions précédentes, en déduire la tension d'alimentation du robot.
\end{question}
    \vspace{\stretch{1}}

\begin{question}
  Quelle est la capacité d'une pile AA. A partir du branchement proposé dans les questions précédentes, en déduire la capacité du système d'alimentation du \textit{mbot}
\end{question}
        \vspace{\stretch{1}}

\pagebreak
\section{Alimentation par un accumulateur}
Il est possible d'alimenter le robot \textit{mbot} à l'aide d'une batterie d'une tension $U=\SI{3,75}{V}$ et d'une capacité $C_a=\SI{1800}{mAh}$.
\begin{question}
  En considérant que cette batterie est chargée à l'aide d'un courant $I=\SI{500}{mA}$, calculez le temps nécessaire à une charge complète de la batterie.
\end{question}

La batterie est chargée à l'aide d'un chargeur branché sur le réseau.

\begin{center}
  \includegraphics[width=0.7\textwidth]{images/reseau_edf_signal}
\end{center}
\begin{question}
  Donnez le type de ce signal, sa fréquence et son amplitude.
\end{question}
 \vspace{\stretch{1}}
\begin{question}
  Ce signal alimente un transformateur parfait de rapport de transformation $r=100$. Dessinez le signal en sortie de ce transformateur.
\end{question}
\vspace{\stretch{4}}
\end{document}
