\documentclass[10pt,fleqn]{article} % Default font size and left-justified equations
\usepackage{import}
\usepackage[%
    pdftitle={Energie et puissance d'un smartphone},
    pdfauthor={Geoffrey Vaquette}]{hyperref}
\subimport{../../../../style/}{preambule}
%\fichetrue
\fichefalse

\proftrue
%\proffalse

\tdtrue
%\tdfalse

\courstrue
%\coursfalse

\subimport{../../../../style/}{new_style}
\subimport{../../../../style/}{macros_SII}
\subimport{../../../../style/}{preambule_trou}

\usepackage{siunitx}
% -------------------------------------
% Déclaration des titres
% -------------------------------------

\def\discipline{Enseignement \\Technologique \\ Transversal}
\def\xxtete{Enseignement Technologique Transversal}

\def\classe{1 STI2D}
\def\xxnumpartie{Séquence 2}
\def\xxpartie{Energie électrique et puissance d'un smartphone}

\def\xxnumchapitre{Séance 3}
\def\xxchapitre{\hspace{.12cm} Énergies, Puissances et rendement}

\def\xxposongletx{2}
\def\xxposonglettext{1.45}
\def\xxposonglety{23}
\def\xxonglet{Seq. 2 -- TD 3}

\def\xxactivite{Synthèse}
\def\xxauteur{\textsl{Geoffrey Vaquette}}

\def\xxcompetences{%
\textsl{%
\textbf{Savoirs et compétences :}
\begin{itemize}[label=\ding{112},font=\color{ocre}]
\item CO2.1	Identifier les flux et la forme de l'énergie, caractériser ses transformations et/ou modulations et estimer l'efficacité globale d'un système.
\end{itemize}
%
}}

\def\xxfigures{
\begin{center}

\end{center}
}%figues de la page de garde
\def\xxpied{%
Chaine d'énergie -- \xxactivite%
}

%---------------------------------------------------------------------------

\renewcommand{\RemplirTrou}{true}
\title{Le bloc « transmettre » du robot mbot}
\date{}
\begin{document}
\maketitle
\chapterimage{}
%\input{Cours/style/new_pagegarde}
\begin{obj}
  On propose dans cette activité de caractériser le bloc de réduction des moteurs du \textit{mbot}
\end{obj}
\section{Introduction}

\begin{question}
  Localisez un bloc moteur sur le robot \textit{mbot}. Combien en comporte-t-il ?
\end{question}
\vspace{\stretch{1}}
\begin{question}
  Quelle est la fonction du réducteur en sortie du moteur ?
\end{question}
\vspace{\stretch{1}}

Soit le schéma de réduction suivant :
\begin{center}
  \includegraphics[width=.4\textwidth]{images/engrenages_mbot}
\end{center}
On donne $Z_1 = 8$, $Z_2 = 32$, $Z_3 = 9$, $Z_4 = 40$, $Z_5 = 20$, $Z_6 = 29$, $Z_7 = 15$, $Z_8 = 30$
\begin{question}
Comptez le nombre de contacts extérieurs présents dans ce montage.
\end{question}
    \vspace{\stretch{1}}
\begin{question}
  Le sens de rotation en sortie sera-t-il le même que le sens de rotation en entrée ?
\end{question}
    \vspace{\stretch{1}}
\begin{question}
  Calculez les rapports de réduction de chaque contact extérieurs.
\end{question}
\vspace{\stretch{2}}
\begin{question}
  Déduisez-en le rapport de réduction global du système.
\end{question}
\vspace{\stretch{1}}
\end{document}
