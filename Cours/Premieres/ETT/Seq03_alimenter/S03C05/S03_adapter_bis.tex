\documentclass[10pt,fleqn]{article} % Default font size and left-justified equations
\usepackage{import}
\usepackage[%
    pdftitle={Energie et puissance d'un smartphone},
    pdfauthor={Geoffrey Vaquette}]{hyperref}
\subimport{../../../../style/}{preambule}
%\fichetrue
\fichefalse

\proftrue
%\proffalse

\tdtrue
%\tdfalse

\courstrue
%\coursfalse

\subimport{../../../../style/}{new_style}
\subimport{../../../../style/}{macros_SII}
\subimport{../../../../style/}{preambule_trou}

\usepackage{siunitx}
% -------------------------------------
% Déclaration des titres
% -------------------------------------

\def\discipline{Enseignement \\Technologique \\ Transversal}
\def\xxtete{Enseignement Technologique Transversal}

\def\classe{1 STI2D}
\def\xxnumpartie{Séquence 2}
\def\xxpartie{Energie électrique et puissance d'un smartphone}

\def\xxnumchapitre{Séance 3}
\def\xxchapitre{\hspace{.12cm} Énergies, Puissances et rendement}

\def\xxposongletx{2}
\def\xxposonglettext{1.45}
\def\xxposonglety{23}
\def\xxonglet{Seq. 2 -- TD 3}

\def\xxactivite{TD}
\def\xxauteur{\textsl{Geoffrey Vaquette}}

\def\xxcompetences{%
\textsl{%
\textbf{Savoirs et compétences :}
\begin{itemize}[label=\ding{112},font=\color{ocre}]
\item CO2.1	Identifier les flux et la forme de l'énergie, caractériser ses transformations et/ou modulations et estimer l'efficacité globale d'un système.
\end{itemize}
%
}}

\def\xxfigures{
\begin{center}

\end{center}
}%figues de la page de garde
\def\xxpied{%
Energies, Puissances et Rendement \xxactivite%
}

%---------------------------------------------------------------------------

\renewcommand{\RemplirTrou}{true}
\title{Le bloc « adapter » du robot mbot}
\date{}
\begin{document}
\maketitle
\chapterimage{}
%\input{Cours/style/new_pagegarde}
\begin{obj}
  On propose dans cette activité de caractériser le fonctionnement en commutation d'un transistor en vue d'alimenter les moteurs du robot \textit{mbot}
\end{obj}
\section{Introduction}
  \begin{center}
    \includegraphics[width=.7\textwidth]{images/transistors}
  \end{center}
  Quelques données :
  \begin{itemize}
    \item $V_\text{CC} = \SI{6}{V}$, \hspace{3cm} $V_\text{BE} = \SI{0.7}{V}$
    \item $\beta = \num{200}$ \hspace{3cm} ${V_\text{CE}}_\text{sat} = \SI{0.2}{V}$
    \item $R_C = \SI{500}{ohm}$ \hspace{3cm} $R_B = \SI{14700}{ohm}$
  \end{itemize}
\begin{question}
  Ecrivez l'équation de la maille d'entrée (la maille de de gauche).

  \textit{(Appliquez la loi de mailles)}
\end{question}
\vspace{\stretch{1}}
\begin{question}
  Ecrivez l'équation de la maille de sortie (la maille de de droite).

  \textit{(Appliquez la loi de mailles)}
\end{question}
\vspace{\stretch{1}}
\pagebreak
\begin{question}
 lorsque $V_\text{IN} = \SI{0}{V}$, aucun courant ne circule dans la résistance $R_B$. Quel est alors l'état du transistor (bloqué ou passant) ?

 Déduisez-en $V_\text{CE}$ et $V_\text{out}$.
\end{question}
    \vspace{\stretch{2}}
\begin{question}
   lorsque $V_\text{IN} = \SI{5}{V}$, quel est l'état du transistor (bloqué ou passant) ?

   Déduisez-en $V_\text{CE}$ et $V_\text{out}$.
\end{question}
    \vspace{\stretch{2}}

\begin{question}
  Lorsque le transistor est passant, utilisez la loi des mailles d'entrée pour déterminer le courant $I_B$
\end{question}
    \vspace{\stretch{1}}

\begin{question}
  Lorsque le transistor est passant, utilisez la loi des mailles de sortie pour déterminer le courant $I_C$
\end{question}
    \vspace{\stretch{1}}

\end{document}
