\documentclass[10pt,fleqn]{article} % Default font size and left-justified equations
\usepackage{import}
\usepackage[%
    pdftitle={Energie et puissance d'un smartphone},
    pdfauthor={Geoffrey Vaquette}]{hyperref}
\subimport{../../../../style/}{preambule}
%\fichetrue
\fichefalse

\proftrue
%\proffalse

\tdtrue
%\tdfalse

\courstrue
%\coursfalse

\subimport{../../../../style/}{new_style}
\subimport{../../../../style/}{macros_SII}
\subimport{../../../../style/}{preambule_trou}

\usepackage{siunitx}
% -------------------------------------
% Déclaration des titres
% -------------------------------------

\def\discipline{Enseignement \\Technologique \\ Transversal}
\def\xxtete{Enseignement Technologique Transversal}

\def\classe{1 STI2D}
\def\xxnumpartie{Séquence 2}
\def\xxpartie{Energie électrique et puissance d'un smartphone}

\def\xxnumchapitre{Séance 3}
\def\xxchapitre{\hspace{.12cm} Chaine d'énergie}

\def\xxposongletx{2}
\def\xxposonglettext{1.45}
\def\xxposonglety{23}
\def\xxonglet{Seq. 2 -- TD 3}

\def\xxactivite{Synthèse}
\def\xxauteur{\textsl{Geoffrey Vaquette}}

\def\xxcompetences{%
\textsl{%
\textbf{Savoirs et compétences :}
\begin{itemize}[label=\ding{112},font=\color{ocre}]
\item CO2.1	Identifier les flux et la forme de l'énergie, caractériser ses transformations et/ou modulations et estimer l'efficacité globale d'un système.
\end{itemize}
%
}}

\def\xxfigures{
\begin{center}

\end{center}
}%figues de la page de garde
\def\xxpied{%
Chaine d'énergie -- \xxactivite%
}

%---------------------------------------------------------------------------

\renewcommand{\RemplirTrou}{true}
\title{Le bloc «convertir» du robot mbot}
\date{}
\begin{document}
\maketitle
\chapterimage{}
%\input{Cours/style/new_pagegarde}
\begin{obj}
  L'objectif de cette activité est de caractériser le moteur du robot \textbf{mbot}.
\end{obj}
\section{Introduction}
Vous avez à votre disposition le manuel d'utilisation du robot \textit{mbot}. Ce document est le support pour les questions suivantes.

La documentation nous informe que la tension nominale du moteur est de $U=\SI{6}{V}$ et que sa vitesse nominale de rotation est de $\Omega=\SI{200}{rpm}$.

On rappelle que la vitesse de rotation de l'arbre d'un moteur à courant continu est proportionnel : $\Omega = K\times U$.
\begin{question}
   Calculez sa vitesse de rotation pour une alimentation de $U_2=\SI{4}{V}$
\end{question}
\vspace{\stretch{1}}
\begin{question}
  Rappelez le principe de fonctionnement d'un moteur à courant continu.
\end{question}
\vspace{\stretch{2}}
\section{Commande par un signal carré}
\begin{center}
  \includegraphics[width=.6\textwidth]{images/carre75}
\end{center}

\begin{question}
  Donnez la période, la fréquence, l'amplitude ainsi que le rapport cyclique du signal ci-dessus.
\end{question}
    \vspace{\stretch{1}}
      \pagebreak
\begin{question}
  Calculez la valeur moyenne $U_\text{moy}$ de ce signal.

  \textit{Rappel : Sur une période, calculez l'aire sous la courbe et divisez par la période.}

\end{question}
    \vspace{\stretch{1}}

\begin{question}
  On considère alors que le moteur est alimenté par une tension continue $U_\text{moy}$. Quelle est sa vitesse de rotation ?
\end{question}
    \vspace{\stretch{1}}

\begin{question}
  Tracez la signal carré d'une amplitude de \SI{6}{V} et d'une fréquence de \SI{20}{kHz} qu'il faudrait alimenter pour alimenter le moteur avec une valeur moyenne de \SI{3}{V}
\end{question}
        \vspace{\stretch{3}}

\end{document}
