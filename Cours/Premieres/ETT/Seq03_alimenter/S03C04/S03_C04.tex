\documentclass[10pt,fleqn]{article} % Default font size and left-justified equations
\usepackage{import}
\usepackage[%
    pdftitle={Energie et puissance d'un smartphone},
    pdfauthor={Geoffrey Vaquette}]{hyperref}
\subimport{../../../../style/}{preambule}
%\fichetrue
\fichefalse

\proftrue
%\proffalse

\tdtrue
%\tdfalse

\courstrue
%\coursfalse

\subimport{../../../../style/}{new_style}
\subimport{../../../../style/}{macros_SII}
\subimport{../../../../style/}{preambule_trou}

\usepackage{siunitx}
% -------------------------------------
% Déclaration des titres
% -------------------------------------

\def\discipline{Enseignement \\Technologique \\ Transversal}
\def\xxtete{Enseignement Technologique Transversal}

\def\classe{1 STI2D}
\def\xxnumpartie{Séquence 2}
\def\xxpartie{Energie électrique et puissance d'un smartphone}

\def\xxnumchapitre{Séance 3}
\def\xxchapitre{\hspace{.12cm} Énergies, Puissances et rendement}

\def\xxposongletx{2}
\def\xxposonglettext{1.45}
\def\xxposonglety{23}
\def\xxonglet{Seq. 2 -- TD 3}

\def\xxactivite{TD}
\def\xxauteur{\textsl{Geoffrey Vaquette}}

\def\xxcompetences{%
\textsl{%
\textbf{Savoirs et compétences :}
\begin{itemize}[label=\ding{112},font=\color{ocre}]
\item CO2.1	Identifier les flux et la forme de l'énergie, caractériser ses transformations et/ou modulations et estimer l'efficacité globale d'un système.
\end{itemize}
%
}}

\def\xxfigures{
\begin{center}

\end{center}
}%figues de la page de garde
\def\xxpied{%
Energies, Puissances et Rendement \xxactivite%
}

%---------------------------------------------------------------------------

\renewcommand{\RemplirTrou}{true}
\begin{document}
\chapterimage{}
%\input{Cours/style/new_pagegarde}
\section{Visionnage d'une vidéo}
\begin{enumerate}
    \item   Quel système est représenté dans cette vidéo ?
    \vspace{\stretch{1}}

    \item  A quoi sert-il ?
    \vspace{\stretch{1}}

    \item  De quel type est l'énergie en entrée du système ?
      \vspace{\stretch{1}}
    \item  De quel type est l'énergie en sortie du système ?
        \vspace{\stretch{1}}
    \item Comment l'énergie est-elle transmise du canal d'eau vers l'axe final ?
        \vspace{\stretch{1}}
    \item L'axe en sortie du système tourne-t-il à la même vitesse, plus vite ou moins vite que l'axe en entrée du système ?
        \vspace{\stretch{1}}
\end{enumerate}

\end{document}
