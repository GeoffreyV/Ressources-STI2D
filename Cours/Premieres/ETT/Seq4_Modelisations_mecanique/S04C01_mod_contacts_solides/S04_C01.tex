\documentclass[10pt,fleqn]{article} % Default font size and left-justified equations
\usepackage[%
    pdftitle={Contacts et mobilités},
    pdfauthor={Geoffrey Vaquette}]{hyperref}
    \usepackage{import}
\subimport{../../../../style/}{preambule.tex}
\usepackage{subcaption}

%\fichetrue
\fichefalse
\proftrue
%\proffalse
%\tdtrue
\tdfalse
\courstrue
%\coursfalse
\subimport{../../../../style/}{new_style}
\subimport{../../../../style/}{macros_SII}
\subimport{../../../../style/}{preambule_trou.tex}

\usepackage{siunitx}
% -------------------------------------
% Déclaration des titres
% -------------------------------------

\def\discipline{Enseignement \\Technologique \\ Transversal}
\def\xxtete{Enseignement Technologique Transversal}

\def\classe{1 STI2D}
\def\xxnumpartie{Seq 4}
\def\xxpartie{Modéliser les interactions entre solides}

\def\xxnumchapitre{Séance 1}
\def\xxchapitre{\hspace{.12cm} Contacts et mobilités}

\def\xxposongletx{2}
\def\xxposonglettext{1.45}
\def\xxposonglety{23}
\def\xxonglet{Seq. 4 -- Se. 1}

\def\xxactivite{Cours}
\def\xxauteur{\textsl{Geoffrey Vaquette}}

\def\xxcompetences{%
\textsl{%
\textbf{Compétences visées :}
\begin{itemize}[label=\ding{112},font=\color{ocre}]
\item \textbf{CO5.1} Expliquer des éléments d’une modélisation proposée relative au comportement de tout ou partie d’un système
\item \textbf{CO5.2} Identifier des variables internes et externes utiles à une modélisation, simuler et valider le comportement du modèle
\end{itemize}
\textbf{Connaissances abordées dans ce cours : }
\begin{itemize}[label=\ding{112},font=\color{ocre}]
\item 3.1.2 Typologie des solutions constructives des liaisons entre solides
\begin{itemize}
  \item Caractérisation des liaisons sur les systèmes
\end{itemize}
\end{itemize}
%
}}

\def\xxfigures{
\begin{center}
\includegraphics[width=3cm]{images/lineique_ponctuel} \\
\includegraphics[width=3cm]{images/2lineique_1ponctuel} \\
% \includegraphics[width=2cm]{images/prise.png} \\
\end{center}
}%figues de la page de garde
\def\xxpied{%
La fonction alimenter/stocker \xxactivite%
}

%---------------------------------------------------------------------------

\renewcommand{\RemplirTrou}{true}
\begin{document}
\chapterimage{images/bandeau.jpg}
\subimport{../../../../style/}{new_pagegarde}
\section*{Introduction}
Pour modéliser l'interaction entre les pièces d'un produit, la première chose à observer est le type de contacts qu'il existe entre ces pièces.

Dans ce cours, nous simplifierons le problème en nous interessant aux contacts entre des surfaces et volumes très simples. Les trois formes que nous considérerons sont \textbf{le plan}, \textbf{le cylindre} et \textbf{la sphère}, représentés sur la Figure~\ref{fig:volumes}.

\begin{figure}[h]
  \centering
  \begin{subfigure}[b]{0.3\textwidth}
    \centering
    \includegraphics[width=0.9\textwidth,height=.1\textheight,keepaspectratio]{images/plan_rouge_seul}
    \caption{\trou{Plan}}
  \end{subfigure}\hfill
  \begin{subfigure}[b]{.3\textwidth}
    \centering
    \includegraphics[width=0.9\textwidth,height=.1\textheight,keepaspectratio]{images/cylindre_rouge_seul}
    \caption{\trou{Cylindre}}
  \end{subfigure}\hfill
  \begin{subfigure}[b]{.3\textwidth}
    \centering
    \includegraphics[width=0.9\textwidth,height=.1\textheight,keepaspectratio]{images/sphere_rouge_seul}
    \caption{\trou{Sphere}}
  \end{subfigure}\hfill
  \caption{Surfaces et volumes utilisés dans ce cours}
  \label{fig:volumes}
\end{figure}


\section{Différents types de contacts}
\begin{aretenir}
  À l'aide des différentes formes de la Figure~\ref{fig:volumes}, on peut définir trois types de contacts : les contacts \trou{surfaciques}, \trou{linéiques} et \trou{ponctuels}.
\end{aretenir}

\begin{remark}
  Pour mettre en évidence ces contacts, on peut imaginer plonger une des pièces dans de la peinture, mettre en contact les deux pièces puis observer la trace laissée sur la pièce non peinte.
\end{remark}

Un exemple de chaque type de contact est donné sur la Figure~\ref{fig:type_contact}.

\begin{figure}[h]
  \begin{subfigure}[b]{0.3\textwidth}
    \centering
    \includegraphics[width=0.9\textwidth,height=.2\textheight,keepaspectratio]{images/plan-plan}
    \caption{\trou{Contact surfacique}}
  \end{subfigure}\hfill
  \begin{subfigure}[b]{0.3\textwidth}
    \centering
    \includegraphics[width=0.9\textwidth,height=.2\textheight,keepaspectratio]{images/cylindre-plan}
    \caption{\trou{Contact linéique}}
  \end{subfigure}\hfill
  \begin{subfigure}[b]{0.3\textwidth}
    \centering
    \includegraphics[width=0.9\textwidth,height=.2\textheight,keepaspectratio]{images/sphere-plan}
    \caption{\trou{Contact ponctuel}}
  \end{subfigure}\hfill
  \caption{Un exemple de chaque type de contact}
  \label{fig:type_contact}
\end{figure}
\subsection{Les contacts surfaciques}

\begin{defi}
  On dit qu'un contact est surfacique lorsque la zone de contact entre deux pièces est une surface non nulle (on peut calculer l'air de cette surface).
\end{defi}

\begin{warn}
  Une surface n'est pas forcément plane (plate).
\end{warn}

Dans ce cours, nous retiendrons trois types de contacts surfaciques, représentés sur la Figure~\ref{fig:surfacique}

\begin{figure}[h]
  \begin{subfigure}[b]{0.3\textwidth}
    \centering
    \includegraphics[width=0.9\textwidth,height=.15\textheight,keepaspectratio]{images/surface_plan}
    \caption{Plan sur plan}
  \end{subfigure}\hfill
  \begin{subfigure}[b]{0.3\textwidth}
    \centering
    \includegraphics[width=0.9\textwidth,height=.15\textheight,keepaspectratio]{images/surface_cylindre}
    \caption{Cylindre sur plan}
  \end{subfigure}\hfill
  \begin{subfigure}[b]{0.3\textwidth}
    \centering
    \includegraphics[width=0.9\textwidth,height=.15\textheight,keepaspectratio]{images/surface_sphere}
    \caption{Sphère dans sphère}
  \end{subfigure}\hfill
  \caption{Trois types de contacts surfaciques}
  \label{fig:surfacique}
\end{figure}

\subsection{Contacts linéiques}
\begin{defi}
  On dit qu'un contact est linéique lorsque la zone de contact entre deux pièces est une ligne. Cette ligne peut être \trou{rectiligne} (une droite) ou \trou{annulaire} (circulaire).
\end{defi}

\begin{figure}[h]
  \begin{subfigure}[b]{0.5\textwidth}
    \centering
    \includegraphics[width=0.9\textwidth,height=.2\textheight,keepaspectratio]{images/lineaire_rectiligne}
    \caption{Contact linéaire \trou{rectiligne}}
  \end{subfigure}\hfill
  \begin{subfigure}[b]{0.5\textwidth}
    \centering
    \includegraphics[width=0.9\textwidth,height=.2\textheight,keepaspectratio]{images/lineaire_annulaire}
    \caption{Contact linéaire \trou{annulaire}}
  \end{subfigure}\hfill
  \caption{Contacts linéiques}
  \label{fig:surfacique}
\end{figure}

  \subsection{Les contacts ponctuels}
  \begin{defi}
    On dit qu'un contact est ponctuel lorsque la zone de contact entre deux pièces est un point.
  \end{defi}
\begin{figure}[h]
  \centering
  \includegraphics[width=0.9\textwidth,height=.2\textheight,keepaspectratio]{images/ponctuel}
  \caption{Contact ponctuel}
  \label{fig:ponctuel}
\end{figure}
\pagebreak
\section{Exemples de contacts entre pièces}
\begin{exemple}
  Prenons l'exemple de la pièce en Figure~\ref{fig:exemple1}, la pièce rouge a deux contacts avec la pièce bleu. Ces contacts sont :

  \begin{itemize}
    \item Un contact \trou{linéique rectiligne}
    \item Un contact \trou{ponctuel}
  \end{itemize}
\end{exemple}
\begin{figure}[h]
  \centering
  \includegraphics[width=0.9\textwidth,height=.2\textheight,keepaspectratio]{images/lineique_ponctuel}
  \caption{Exemple 1}
  \label{fig:exemple1}
\end{figure}

\begin{exemple}
  Prenons l'exemple de la pièce en Figure~\ref{fig:exemple2}, la pièce rouge a \trou{trois} contacts avec la pièce bleu. Ces contacts sont :

  \begin{itemize}
    \item \trou{Deux contacts linéiques rectilignes}
    \item \trou{Un contact ponctuel}
  \end{itemize}
\end{exemple}
\begin{figure}[h]
  \centering
  \includegraphics[width=0.9\textwidth,height=.2\textheight,keepaspectratio]{images/2lineique_1ponctuel}
  \caption{Exemple 2}
  \label{fig:exemple2}
\end{figure}

\begin{remark}
  Les exercices utilisés durant cette séance sont tirés du site suivant. Vous trouverez sur cette page des exercices pour vous entrainer à reconnaître les contacts entre pièces.\\
  \url{http://www.ecligne.net/mecanique/1_modelisation/1_les_contacts/sommaire.html}
\end{remark}

\end{document}
