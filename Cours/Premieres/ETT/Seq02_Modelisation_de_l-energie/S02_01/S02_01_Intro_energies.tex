\documentclass[a4paper]{article}

\usepackage{lmodern}
\usepackage{geometry}
\usepackage{calc}
\usepackage{xcolor}
\usepackage{amsmath,amssymb,mathrsfs}
\usepackage{ifthen}

\usepackage[francais]{babel}
\usepackage[utf8]{inputenc}
\usepackage[T1]{fontenc}
\usepackage{amsmath}
\usepackage{graphicx}
\usepackage[colorinlistoftodos]{todonotes}

\usepackage{listings}
\usepackage{color}


\usepackage{pifont}% http://ctan.org/pkg/pifont
\newcommand{\cmark}{\color{green}\ding{51}}%
\newcommand{\xmark}{\color{red}\ding{55}}%
\newcommand{\fmark}{\ding{229}}%
\newcommand{\itemc}{\item[\cmark]}%
\newcommand{\itemx}{\item[\xmark]}%
\newcommand{\itemf}{\item[\fmark]}%

%% Ça, ce sont les commandes publiques %%
\newcommand{\RemplirTrou}{false}
\newcommand{\TailleTrou}{\Large}
\newcommand{\ColorTrou}{black}
\newlength{\DeltaTrou}
\setlength{\DeltaTrou}{0pt}
%%%%%%%%%%%%%%%%%%%%%%%%%%%%%%%%%%%%%%%%%

\newsavebox{\totoTrou}
\newsavebox{\tataTrou}
\newlength{\longpointTrou}
\newlength{\profondeurTrou}
\newlength{\hauteurTrou}


\newcommand{\trou}[2][\RemplirTrou]{%
    \savebox{\tataTrou}{\hbox{\color{\ColorTrou}.}}%
    \leavevmode%
    \ifthenelse{\equal{#1}{false}}{%
        \ifmmode%
            \ifinner%
                \savebox{\totoTrou}{\TailleTrou$#2$}%
                \settodepth{\profondeurTrou}{\TailleTrou$#2$}%
                \settoheight{\hauteurTrou}{\TailleTrou$#2$}%
            \else%
                \savebox{\totoTrou}{\TailleTrou$\displaystyle#2$}%
                \settodepth{\profondeurTrou}{\TailleTrou$\displaystyle#2$}%
                \settoheight{\hauteurTrou}{\TailleTrou$\displaystyle#2$}%
            \fi%
        \else%
            \savebox{\totoTrou}{\TailleTrou#2}%
                \settodepth{\profondeurTrou}{\TailleTrou#2}%
                \settoheight{\hauteurTrou}{\TailleTrou#2}%
        \fi%
        \setlength{\longpointTrou}{0pt}%
        \whiledo{\longpointTrou<\wd\totoTrou}{%
            \hbox{%\TailleTrou%\rule{0pt}{\baselineskip}% <-- version d'origine
                  \rule[-\profondeurTrou-\DeltaTrou]{0pt}{\profondeurTrou+\hauteurTrou+2\DeltaTrou}% <-- ma version
                  \usebox{\tataTrou}%
            }% fin de \hbox
            \penalty0\advance\longpointTrou by\wd\tataTrou%
        }% fin du whiledo
    }% fin du then de ifthenelse
    {%
        {\textbf{\color{red}#2}}%
    }% fin du else de ifthenelse
}% fin de la définition de \trou


% Un peu de mise en page
\setlength{\parindent}{0pt}
\geometry{scale=0.8}
%\renewcommand{\baselinestretch}{1.5} % <--- Si on changer la taille des interlignes


%% Les commandes publiques de PreambuleTrou.tex

%%---Commande 1
\renewcommand{\RemplirTrou}{true}
% Par défaut \RemplirTrou est déjà sur "false" donc cette ligne ne sert à rien.
% Si on remplace le false par "true" (ou n'importe quoi qui n'est pas "true")
% alors les trous seront remplacés par leur contenu. Ça permet de basculer
% de la version "trouée" à la version "corrigée"

%%---Commande 2
\renewcommand{\TailleTrou}{\huge}
% Par défaut \TailleTrou est déjà sur "\Large" donc cette ligne ne sert à rien.
% Mais on peut remplacer "\Large" par "\normalsize", "\large", "\LARGE", "\huge"
% ou "\Huge". Le trou aura la même taille que "son-contenu-caché-mis-en-taille-\TailleTrou".
% La raison est, qu'en général, un élève écrit plus gros que n'écrit LaTeX.
% Attention, cette taille n'intervient que lorsqu'il y un trou. Si \RemplirTrou est
% sur "true" alors le contenu prend la taille du texte courant.

%%---Commande 3
\setlength{\DeltaTrou}{2pt}
% Par défaut \DeltaTrou est déjà sur "0pt" donc cette ligne ne sert à rien.
% Mais si on met "5pt" par exemple, ça augmente la *hauteur* des trous de
% "5pt". C'est au cas où l'utilisateur ne trouve pas son bonheur avec \TailleTrou
% et veuille ajuster la hauteur en valeur absolue. (La valeur peut être < 0).

%%---Commande 4
\renewcommand{\ColorTrou}{black}
% Par défaut \ColorTrou est déjà sur "black" donc cette ligne ne sert à rien.
% On peut même "gray", "green" etc. En revanche, cette couleur sera valable
% aussi bien avec \RemplirTrou sur "false" que sur "true".

% RQ : toutes ces commandes peuvent être modifiées en cours de route. Voir ci-dessous
%      un exemple.


\title{Energie et puissance -- Rappel sur les conversions d'unité}

\author{Lycée Louis Armand -- G. Vaquette}

\date{ETT -- 1 STI2D -- Octobre 2018}

\newcommand{\parcoeur}{{\large\ding{164}} }
\newcommand{\afaire}{{\large\ding{46}} }
\newcommand{\correction}{{\large\ding{52}} }
\begin{document}
\maketitle

%\begin{abstract}

%\end{abstract}

Légende : \begin{itemize}
    \item[\afaire] A compléter seul, en classe
    \item[\correction] Correction écrite en classe
    \item[\parcoeur] A connaître ABSOLUMENT \textbf{par coeur} !
\end{itemize}
\section{Définitions}
\subsection{Energie}
\ding{46} Pour vous, qu'est ce que l'énergie ? 

\trou{L'énergie est la capacité d'un système à modifier un état (mettre en mouvement, produire un rayonnement).}

\textbf{Définition après restitution : }\trou{L'énergie est la capacité d'un système à modifier un état (mettre en mouvement, produire un rayonnement, de la chaleur). }

\ding{46} A votre avis, peut-on créer de l'énergie ? Si oui, comment ? Sinon, expliquez.

\trou{Lorem ipsum dolor sit amet, consectetur adipiscing elit. Duis sollicitudin finibus nulla. Sed interdum nisl aliquam nulla pharetra, utluctus metus accumsan. In tellus maurise}

\textbf{Citation de Antoine de Lavoisier : }
\trou{Rien ne se perd, rien ne se crée, tout se transforme.}

\trou{Cela signifie que la quantité d'énergie dans l'univers est constante.}

\paragraph{\parcoeur Notations et unités : }En physique, on note l'énergie \trou{E}, elle s'exprime en (son unité :) \trou{Joule}. 

\paragraph{Autres unités : }
\trou{E en calories : $ 1\text{cal} = 4.18\text{J}$}

\trou{E en tonne équivalent pétrole : $ 1\text{tep} = 419\text{GJ}$}

\trou{E en Watt Heure : $ 1\text{Wh} = 3600\text{J}$}



\pagebreak
%% ==== ==== La puissance ==== ==== %%
\subsection{Puissance}
\ding{46} Pour vous, qu'est ce qu'une puissance ? 

\trou{La puissance, c'est de l'énergie par unité de temps. On peut aussi dire \textit{un débit d'énergie}}

\paragraph{Définition après restitution : }
\trou{La puissance, c'est de l'énergie par unité de temps. On peut aussi dire \textit{un débit d'énergie}.}
\paragraph{Notations et unités : }En physique, on note la puissance \trou{P}, elle s'exprime en (son unité :) \trou{Watt (W)}. 

\textbf{Exemple : }\trou{La puissance, c'est de l'énergie par unité de temps. On peut aussi dire \textit{un débit d'énergie}}

\paragraph{\parcoeur}Equation liant l'énergie à la puissance : \begin{align}
    P = \frac{E}{\Delta t} &\hspace{1cm} 
    E=P\times \Delta t
\end{align}

\section{Rappel sur les conversion d'unités : }
\pagebreak
\section{Quelques exemples : }
Imaginons une toute petite pile qui contiendrait 7 200 J. Si cette énergie est consommée en 1h par une toute petite ampoule. quelle était la puissance de cette ampoule ? \\
\afaire \trou{L'élève travaille : La puissance et une énergie par unité de temps : $P = \frac{E}{\Delta t} = \frac{7200}{3600} = 2 W $}

\correction \trou{La puissance et une énergie par unité de temps : $P = \frac{E}{\Delta t} = \frac{7200}{3600} = 2 W $}

\paragraph{}
Cette même pile (contenant 7 200 J) est maintenant consommée en 30 min par une autre ampoule. Quelle est la puissance de cette ampoule ? 

\afaire \trou{L'élève travaille : La puissance et une énergie par unité de temps : $P = \frac{E}{\Delta t} = \frac{7200}{1800} = 4 W $}

\correction \trou{$P = \frac{E}{\Delta t} = \frac{7200}{1800} = 4 W $ Cette lampe a donc une puissance de 4 W.}

\paragraph{} Une machine à laver AAA a une puissance de 2.0 kW. Imaginons un foyer dans lequel elle est utilisée 3 fois par semaine pendant 1h30.  
\begin{itemize}
    \item Quelle énergie consomme cette machine pendant 1 utilisation de 1h30 ? En J, en Wh, en cal, en tep ? 
    \item Quelle énergie consomme cette machine sur un mois ? 
\end{itemize}

\afaire \trou{L'élève travaille : La machine consomme 2000 x 1.5 x 3 x 4 = beaucoup. La machine consomme 2000 x 1.5 x 3 x 4 = beaucoup. La machine consomme 2000 x 1.5 x 3 x 4 = beaucoup. Et en joule ca donne :  Ha ouais quand même !!! Et voilà de la place supplémentaire pour l'élève oui }
\vspace{1cm}

\correction \trou{$E = {P}\times {\Delta t} = 2000 \times (1.5\times 3600) =  10.8MJ = 3000Wh$ 
Sur un mois, cette machine consomme alors $E = 3\times 4 \times 3000 = 36 000 Wh$. \textit{Et voilà de la place supplémentaire pour l'élève qui a besoin de place pour écrire. Mais aussi pour aller à la ligne, etc.}}
\end{document}