\documentclass[10pt,fleqn]{article} % Default font size and left-justified equations
\usepackage{import}
\usepackage[%
    pdftitle={Energie et puissance d'un smartphone},
    pdfauthor={Geoffrey Vaquette}]{hyperref}
\subimport{../../../../style/}{preambule}
%\fichetrue
\fichefalse

\proftrue
%\proffalse

\tdtrue
%\tdfalse

\courstrue
%\coursfalse

\subimport{../../../../style/}{new_style}
\subimport{../../../../style/}{macros_SII}
\subimport{../../../../style/}{preambule_trou}

\usepackage{siunitx}
% -------------------------------------
% Déclaration des titres
% -------------------------------------

\def\discipline{Enseignement \\Technologique \\ Transversal}
\def\xxtete{Enseignement Technologique Transversal}

\def\classe{1 STI2D}
\def\xxnumpartie{Séquence 2}
\def\xxpartie{Energie électrique et puissance d'un smartphone}

\def\xxnumchapitre{Séance 3}
\def\xxchapitre{\hspace{.12cm} Énergies, Puissances et rendement}

\def\xxposongletx{2}
\def\xxposonglettext{1.45}
\def\xxposonglety{23}
\def\xxonglet{Seq. 2 -- DS 3}

\def\xxactivite{DS}
\def\xxauteur{\textsl{Geoffrey Vaquette}}

\def\xxcompetences{%
\textsl{%
\textbf{Savoirs et compétences :}
\begin{itemize}[label=\ding{112},font=\color{ocre}]
\item CO2.1	Identifier les flux et la forme de l'énergie, caractériser ses transformations et/ou modulations et estimer l'efficacité globale d'un système.
\end{itemize}
%
}}

\def\xxfigures{
\begin{center}
\includegraphics[height=4cm]{images/smartphone.png} \\
\includegraphics[height=4cm]{images/petrole.png} \\
\includegraphics[height=4cm]{images/gaz.png} \\
\end{center}
}%figues de la page de garde
\def\xxpied{%
Energies, Puissances et Rendement \xxactivite%
}

%---------------------------------------------------------------------------

\renewcommand{\RemplirTrou}{true}
\begin{document}
\chapterimage{png/Fond_solaire}

\begin{obj}
Déterminer l’énergie et la puissance disponibles dans un système

L’étude suivante concerne un smartphone. 

Vous serez amené à
 calculer l’énergie présente dans la batterie de ce téléphone ainsi que les puissances et énergies nécessaires
 nécessaires pour différentes utilisations de celui-ci.
 
\end{obj}
\section{Étude énergétique d'un iPhone X}
\subsection{Quelques données}
Caractéristiques de la batterie :
\begin{itemize}
 \item Technologie : Li-Ion (lithium-ion)
\item Capacité : $ C = \SI{2717}{mAh} $
\item Tension : $U=\SI{3.81}{V}$
\item Autonomie en utilisation : 275 min
\end{itemize}



\begin{exercise}~

\begin{question}
  Quelle solution permet de remplir la fonction « Alimenter / Stocker » dans le cas de ce smartphone ?
\end{question} 
\begin{solution}
  La fonction alimenter/stocker est réalisée par la batterie du smartphone. 
\end{solution}

\begin{question}
  Calculer l’énergie électrique $\omega_\text{bat}$que contient la batterie.
\end{question}
\begin{solution}
  On connaît la tension de la batterie, ainsi que sa capacité. L'énergie 
  contenue dans la batterie est $$\omega_\text{bat} = U\times C = 3.81 \times \num{2717e-3}=\SI{10.35}{Wh} $$
\end{solution}

\begin{question}
  A partir des données annoncées, calculer la puissance de ce Smartphone en 
 utilisation
\end{question}
\begin{solution}
La puissance moyenne est l'énergie consommé divisée par le temps qu'il a fallu pour consommer cette énergie : $P = \frac{\omega_\text{bat}}{t}$ 
$$P_{\text{utilisation}} = \frac{10.35}{275/60} = \SI{2.26}{W}$$
\end{solution}

\begin{question}
  En déduire le courant consommé en utilisation. 
\end{question}
\begin{solution}
  Connaissant la relation liant la puissance, le courant et la tension 
  $P=U\times I$, on déduit $I = \frac{P}{U}$.
  $$I_{\text{utilisation}} = \frac{2.26}{3.81} = \SI{593}{mA}$$
\end{solution}

\begin{question}
  En supposant qu’une charge complète de la batterie doit être effectuée tous les jours, déterminer l’énergie électrique $E_{\text{tel annuel}}$ consommée par le téléphone en une année (en Wh et en J)
\end{question}
\begin{solution}
  Tous les jours, le Smartphone consomme une énergie de 13.2 Wh.
Il faudra le recharger 365 fois en un an.
$$E_{\text{tel annuel}} = 365 \times 10.35 = \SI{3 777}{Wh} = \SI{3.8}{kWh}$$
\end{solution}

\begin{question}
  En supposant le rendement d'un chargeur de téléphone de $\eta = 0.95$, calculez l'énergie annuelle consommée sur le réseau par un téléphone en France.
  
  On rappelle que $E_s = \eta \times E_e$ avec $\eta$ le rendement, $E_s$ l'énergie en \textbf{sortie} d'un système et $E_e$ l'énergie en \textbf{entrée} d'un système. 
\end{question}
\begin{solution}
  On cherche l'énergie $E_{\text{elec}}$ en entrée du chargeur (nécessaire à recharger le téléphone) en connaissant l'énergie en sortie (énergie consommée par le téléphone). On a donc $ E_{\text{elec}} = \frac{E_{\text{tel annuel}}}{\eta} = \SI{3971}{Wh}$
\end{solution}

\begin{question}
Calculez l'énergie perdue (énergie non-utile) annuellement pour la recharge de téléphones portables. Sous quelle forme est perdue cette énergie ? 
\end{question}
\begin{solution}
  Cette énergie est perdue sous forme de chaleur. On perd $E_{\text{perdue}} = E_{\text{elec}} * (1-\eta) $
\end{solution}

\begin{question}
  Les statistiques indiquent qu'en 2017, 73\% des français de plus de 12 ans possédaient un smartphone. En supposant que tous les français ait un portable équivalent à l'iPhone X en consommation d'énergie, Calculez l'énergie totale que représente la recharge des téléphone en France, en 2017. 
  
  \textit{On suppose que 59 Millions de Français avaient plus de 12 ans en 2017. }
\end{question}
\begin{solution}
  $E_{totale} = \num{59e6}\times 0.73 \times E_{elec}  = \SI{171.2}{GWh}$
\end{solution}

\begin{question}
Convertissez cette dernière valeur en joules (J) et en Tep. Rappel : $\SI{1}{Tep} = \SI{4.187e10}{J}$
\end{question}

\begin{solution}
  En joules : On divise le résultat en Wh par 3600
  En Tep : On divise le résultat en J par \num{4.187e10}
\end{solution}
\end{exercise}


\end{document}