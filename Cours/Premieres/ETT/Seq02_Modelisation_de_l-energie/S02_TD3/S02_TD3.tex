\documentclass[10pt,fleqn]{article} % Default font size and left-justified equations
\usepackage{import}
\usepackage[%
    pdftitle={Energie et puissance d'un smartphone},
    pdfauthor={Geoffrey Vaquette}]{hyperref}
\subimport{../../../../style/}{preambule}
%\fichetrue
\fichefalse

\proftrue
%\proffalse

\tdtrue
%\tdfalse

\courstrue
%\coursfalse

\subimport{../../../../style/}{new_style}
\subimport{../../../../style/}{macros_SII}
\subimport{../../../../style/}{preambule_trou}

\usepackage{siunitx}
% -------------------------------------
% Déclaration des titres
% -------------------------------------

\def\discipline{Enseignement \\Technologique \\ Transversal}
\def\xxtete{Enseignement Technologique Transversal}

\def\classe{1 STI2D}
\def\xxnumpartie{Séquence 2}
\def\xxpartie{Energie électrique et puissance d'un smartphone}

\def\xxnumchapitre{Séance 3}
\def\xxchapitre{\hspace{.12cm} Énergies, Puissances et rendement}

\def\xxposongletx{2}
\def\xxposonglettext{1.45}
\def\xxposonglety{23}
\def\xxonglet{Seq. 2 -- TD 3}

\def\xxactivite{TD}
\def\xxauteur{\textsl{Geoffrey Vaquette}}

\def\xxcompetences{%
\textsl{%
\textbf{Savoirs et compétences :}
\begin{itemize}[label=\ding{112},font=\color{ocre}]
\item CO2.1	Identifier les flux et la forme de l'énergie, caractériser ses transformations et/ou modulations et estimer l'efficacité globale d'un système.
\end{itemize}
%
}}

\def\xxfigures{
\begin{center}
\includegraphics[height=4cm]{images/smartphone.png} \\
\includegraphics[height=4cm]{images/petrole.png} \\
\includegraphics[height=4cm]{images/gaz.png} \\
\end{center}
}%figues de la page de garde
\def\xxpied{%
Energies, Puissances et Rendement \xxactivite%
}

%---------------------------------------------------------------------------

\renewcommand{\RemplirTrou}{true}
\begin{document}
\chapterimage{png/Fond_solaire}
%\input{Cours/style/new_pagegarde}

\begin{obj}
Analyser et comparer les apports d’énergies de différentes sources. 

À travers cette activité, vous serez amené à comparer une source
d’énergie renouvelable et non renouvelable afin d’en faire ressortir les
avantages et inconvénients de chacune.
\end{obj}
\section{Caractérisation d'un signal périodique}
\begin{exercise}
    Soit un signal rectangulaire ayant les caractéristiques suivantes : 
        
    \begin{itemize}
        \item Fréquence $f = \SI{1}{kHz}$
        \item Rapport cyclique $ = \SI{70}{\%}$
        \item $U_{\text{max}} = \SI{8}{V}$
        \item $U_{\text{min}} = \SI{-3}{V}$
    \end{itemize}
    
    \begin{question}
        Représentez le signal sous la forme d'un chronogramme.
    \end{question}
    \begin{question}
        Calculez la valeur de la période.
    \end{question}
    \begin{question}
        Déterminez la valeur de l'état haut $Th$
    \end{question}
    \begin{question}
        Déterminez la valeur moyenne du signal.
    \end{question}
\end{exercise}


\end{document}