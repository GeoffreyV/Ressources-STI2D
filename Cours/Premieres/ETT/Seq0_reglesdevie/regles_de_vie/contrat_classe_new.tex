%%%%%%%%%%%%%%%%%%%%%%%%%%%%%%%%%%%%%%%%%%%%%%%%
% E.Pinault-Bigeard - e.pinault-bigeard@upsti.fr
% http://s2i.pinault-bigeard.com
% CC BY-NC-SA 2.0 FR - http://creativecommons.org/licenses/by-nc-sa/2.0/fr/
%%%%%%%%%%%%%%%%%%%%%%%%%%%%%%%%%%%%%%%%%%%%%%%%
\documentclass[11pt]{article}
%%%%%%%%%%%%%%%%%%%%%%%%%%%%%%%%%%%%%%%%%%%%%%%%
% Package UPSTI_Document
%%%%%%%%%%%%%%%%%%%%%%%%%%%%%%%%%%%%%%%%%%%%%%%%
\usepackage{subcaption}
\usepackage{UPSTI_Document}

%---------------------------------%
% Paramètres du package
%---------------------------------%

% Version du document (pour la compilation)
% 1: Document prof
% 2: Document élève
% 3: Document à publier
\newcommand{\UPSTIidVersionDocument}{2}

% Variante
%\newcommand{\UPSTIvariante}{2}

% Classe
% 1: PTSI				6: PSI*			11: TSI2		16: Spé
% 2: PT	(par défaut)	7: MPSI			12: ATS
% 3: PT*				8: MP			13: PC
% 4: PCSI				9: MP*			14: PC*
% 5: PSI				10: TSI1		15: Sup
%\newcommand{\UPSTIidClasse}{2}

% Affichage personnalisé de la classe
\newcommand{\UPSTIclasse}{Première STI2D}

% Matière
% 1: S2I (par défaut)    2: IPT     3: TIPE
%\newcommand{\UPSTIidMatiere}{9}

% Type de document
% 0: Custom*				7: Fiche Méthode			14: Document Réponses
% 1: Cours (par défaut)		8: Fiche Synthèse    		15: Programme de colle
% 2: TD     				9: Formulaire
% 3: TP						10: Memo
% 4: Colle					11: Dossier Technique
% 5: DS						12: Dossier Ressource
% 6: DM						13: Concours Blanc
% * Si on met la valeur 0, il faut décommenter la ligne suivante:
%\newcommand{\UPSTItypeDocument}{Custom}
\newcommand{\UPSTIidTypeDocument}{1}

% Titre dans l'en-tête
\newcommand{\UPSTItitreEnTete}{Règles de revient ETT}
%\newcommand{\UPSTItitreEnTetePages}{UPSTItitreEnTetePages}
%\newcommand{\UPSTIsousTitreEnTete}{UPSTIsousTitreEnTete}

% Titre
\newcommand{\UPSTItitrePreambule}{Droits et devoir en ETT }
\newcommand{\UPSTItitre}{}

% Durée de l'activité (pour DS, DM et TP)
%\newcommand{\UPSTIduree}{3h}

% Note de bas de première page
%\newcommand{\UPSTInoteBasDePremierePage}{Note de bas de 1ère page}

% Numéro (ajoute " n°1" après DS ou DM)
%\newcommand{\UPSTInumero}{1}

% Numéro chapitre
%\newcommand{\UPSTInumeroChapitre}{4}

% En-tête customisé
%\newcommand{\UPSTIenTetePrincipalCustom}{UPSTIenTetePrincipalCustom}

% Message sous le titre
%\newcommand{\UPSTImessage}{Message sous le titre}

% Référence au programme
%\newcommand{\UPSTIprogramme}{\EPBComp \EPBCompP{B1-02}, \EPBCompP{B2-49}, \EPBCompS{B2-50}, \EPBCompS{B2-51}, \EPBCompP{C1-07}, \EPBCompP{C1-08}}

% Si l'auteur n'est pas l'auteur par défaut
%\renewcommand{\UPSTIauteur}{UPSTI}

% Si le document est réalisé au nom de l'équipe
%\newcommand{\UPSTIdocumentCollegial}{1}

% Source
%\newcommand{\UPSTIsource}{UPSTI}

% Version du document
\newcommand{\UPSTInumeroVersion}{0.2}

%-----------------------------------------------
\UPSTIcompileVars		% "Compile" les variables
%%%%%%%%%%%%%%%%%%%%%%%%%%%%%%%%%%%%%%%%%%%%%%%%
% Avant d'entrer en classe
% Portable rangé
% Couvre chefs
% Oreillette
%
% Casque moto non visible
% Sac au sol
% Enlever son manteau
%
% En entrant :
% Bonjour
%
% On ne boit pas ni ne mange en classe
%%%%%%%%%%%%%%%%%%%%%%%%%%%%%%%%%%%%%%%%%%%%%%%%
% Début du document
%%%%%%%%%%%%%%%%%%%%%%%%%%%%%%%%%%%%%%%%%%%%%%%%
\begin{document}
\UPSTIbuildPage
\title{Contrat et règles en classe de STI2D}
\date{Année 2018 - 2019}

Pour l'année 2018-2019, chaque élève, ainsi que le professeur s'engage à respecter les règles suivantes :
\section{Respect}
\subsection{Droits}
\begin{itemize}
    \item Chaque élève a le droit au respect des autres élèves ainsi que du professeur.
    \item Le professeur a le droit au respect de chacun des élèves.
\end{itemize}

\subsection{Devoirs}
\begin{itemize}
    \item Chaque individu de la classe s'adresse la parole correctement, poliment, sans vulgarité ni agressivité à l'égard de quiconque.
    \item Lorsqu'un élève s'adresse au professeur et/ou au reste de la classe, tout le monde l'écoute sans l'interrompre.
    \item Lorsque le professeur s'adresse aux élèves, tous les élèves l'écoute.
    %\begin{itemize}
      \item Cela implique l'absence de tout écouteur, casque dans la salle de classe.
      \item Cela implique l'extinction des téléphone portables dès l'entrée en cours.
    %\end{itemize}
\end{itemize}

\section{Travail en classe}
\subsection{Droits}
\begin{itemize}
    \item Tout élève a le droit d'écouter et de participer aux cours, aux TDs, aux TPs.
    \item Tout élève a le droit de poser une question pour demander une clarification d'une notion de cours ou d'une question qu'il n'aurait pas comprise.
    \item Le professeur a le droit d'interroger un élève pour vérifier qu'une notion a bien été comprise.
    \item Le professeur a le droit d'être écouté lorsqu'il s'adresse à la classe.
    \item Les élèves ont le droit de connaître les modalités d'évaluation ainsi que les résultats des évaluations.
\end{itemize}
\subsection{Devoirs}
\begin{itemize}
    \item Pour garantir une répartition de la parole, les élèves lèvent la main s'ils veulent la parole. C'est l'enseignant qui distribue  la parole.
    \item Pour que tout le monde puisse écouter ce qui est dit et travaille dans le calme, un élève qui n'a pas la parole n'est pas autorisé à parler, ni bavarder avec un camarade de classe.
    \item Le professeur doit répartir la parole de façon équitable, tout en garantissant la bonne avancée du cours.
    \item Afin de pouvoir suivre le cours, les élèves ont le devoir de prendre des notes et de recopier le cours, les exercices et les corrigés donnés par le professeur.
    \item L'utilisation du téléphone portable n'est, a-priori, pas autorisée. Si l'élève doit, pour une raison urgente, l'utiliser il doit en demander l'autorisation préalable au professeur.
    \item Le professeur s'engage à tout faire pour corriger les devoirs rendus avant la séance suivante et s'engage à le rendre au plus tard sous 15 jours.
\end{itemize}

\section{Assiduité}
\subsection{Droits}
\begin{itemize}
    \item Les élèves et le professeur ont le droit de sortir à la dernière sonnerie de la séance de cours.
    \item Les élèves et le professeurs ont le droit de commencer le cours à l'heure, sans perturbation extérieure.
\end{itemize}

\subsection{Devoirs}
\begin{itemize}
    \item Afin de débuter le cours le plus rapidement possible, les élèves doivent entrer dans le calme, s'asseoir à leur place et sortir leurs affaires de cours dès que le professeur les invite à entrer en classe.
    \item Pour ne pas perturber le cours, les élèves ne sont plus autorisés à entrer en classe une fois la porte fermée.
    \item En cas d'absence, l'élève doit rattraper le cours sur pronote ou auprès d'un autre élève AVANT la séance suivante.
    \item Le professeur doit mettre sur pronote les documents distribués et utilisés en classe, afin que les élèves puissent y avoir accès.
\end{itemize}

\section{Matériel}
\subsection{Droits}
\begin{itemize}
    \item Tout élève a le droit d'avoir les supports de cours, TD et TP de la séance.
    \item Toute personne a le droit au respect de ses affaires tant qu'il n'en fait pas un usage non autorisé.
    \item L'élève a le choix de l'organisation de ses documents, tant qu'ils sont facilement et rapidement accessibles à tout moment.
    \item Le professeur peut exiger qu'un ou plusieurs élèves apportent leur classeur (ou cahier ou porte-vues) contenant tous les documents de l'année, pour vérifier son organisation et la présence de tous les documents.
    \item S'il considère que la façon de trier et d'organiser les documents d'un élève n'est pas satisfaisante, le professeur est en droit d'exiger une autre organisation.
\end{itemize}
\subsection{Devoirs}
\begin{itemize}
    \item Tout élève doit avoir avec lui de quoi écrire (feuille et stylo) ainsi que l'ensemble des documents de la séquence en cours.
    \item Les élèves doivent conserver, de manière organisée et facilement accessible, la totalité des documents utilisés en cours, jusqu'à la fin de l'année scolaire (et idéalement jusqu'au baccalauréat).
    \item Pour prévenir toute perte de document, l'utilisation d'une pochette pour conserver les documents, n'est pas autorisée.

\end{itemize}

\section{Punition et sanctions}
Tout manquement par un élève à l'un des devoirs cités dans ce document ou dans le règlement intérieur du lycée l'expose à une punition ou une sanction.

Ces punitions ou sanctions peuvent prendre la forme d'une observation dans le carnet de correspondance à faire signer par le référent légal de l'élève, d'heures de retenues avec travail supplémentaire, d'exclusion de cours et/ou de privation d'un droit cité dans ce document.

\subsection{Punitions et sanctions types}
\begin{itemize}
    \item Les bavardages, les amusements en cours seront sanctionnés après un premier avertissement oral.
    \item Le téléphone portable sera systématiquement confisqué s'il est visible en cours. Il sera mené à l'administration dès la deuxième confiscation du trimestre.
    \item L'absence du matériel nécessaire à la prise de note ou l'absence d'un travail donné à la maison sera sanctionnée la première fois par une observation dans le carnet de correspondance. La deuxième fois, l'élève pourra être exclu du cours et/ou se voir attribuer une heure de retenue avec travail supplémentaire.
    \item Tout aliment sera confisqué et entrainera l'exclusion du ou des élèves concernés.
\end{itemize}

\end{document}
