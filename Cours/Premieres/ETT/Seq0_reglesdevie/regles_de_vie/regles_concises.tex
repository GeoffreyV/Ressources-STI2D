%%%%%%%%%%%%%%%%%%%%%%%%%%%%%%%%%%%%%%%%%%%%%%%%
% E.Pinault-Bigeard - e.pinault-bigeard@upsti.fr
% http://s2i.pinault-bigeard.com
% CC BY-NC-SA 2.0 FR - http://creativecommons.org/licenses/by-nc-sa/2.0/fr/
%%%%%%%%%%%%%%%%%%%%%%%%%%%%%%%%%%%%%%%%%%%%%%%%
\documentclass[11pt]{article}
%%%%%%%%%%%%%%%%%%%%%%%%%%%%%%%%%%%%%%%%%%%%%%%%
% Package UPSTI_Document
%%%%%%%%%%%%%%%%%%%%%%%%%%%%%%%%%%%%%%%%%%%%%%%%
\usepackage{subcaption}
\usepackage{UPSTI_Document}

%---------------------------------%
% Paramètres du package
%---------------------------------%

% Version du document (pour la compilation)
% 1: Document prof
% 2: Document élève
% 3: Document à publier
\newcommand{\UPSTIidVersionDocument}{2}

% Variante
%\newcommand{\UPSTIvariante}{2}

% Classe
% 1: PTSI				6: PSI*			11: TSI2		16: Spé
% 2: PT	(par défaut)	7: MPSI			12: ATS
% 3: PT*				8: MP			13: PC
% 4: PCSI				9: MP*			14: PC*
% 5: PSI				10: TSI1		15: Sup
%\newcommand{\UPSTIidClasse}{2}

% Affichage personnalisé de la classe
\newcommand{\UPSTIclasse}{Toute section}

% Matière
% 1: S2I (par défaut)    2: IPT     3: TIPE
\newcommand{\UPSTIidMatiere}{6}

% Type de document
% 0: Custom*				7: Fiche Méthode			14: Document Réponses
% 1: Cours (par défaut)		8: Fiche Synthèse    		15: Programme de colle
% 2: TD     				9: Formulaire
% 3: TP						10: Memo
% 4: Colle					11: Dossier Technique
% 5: DS						12: Dossier Ressource
% 6: DM						13: Concours Blanc
% * Si on met la valeur 0, il faut décommenter la ligne suivante:
\newcommand{\UPSTItypeDocument}{Note}
\newcommand{\UPSTIidTypeDocument}{0}

% Titre dans l'en-tête
\newcommand{\UPSTItitreEnTete}{Texte de sensibilisation}
%\newcommand{\UPSTItitreEnTetePages}{UPSTItitreEnTetePages}
%\newcommand{\UPSTIsousTitreEnTete}{UPSTIsousTitreEnTete}

% Titre
\newcommand{\UPSTItitrePreambule}{À lire en début de séance}
\newcommand{\UPSTItitre}{}

% Durée de l'activité (pour DS, DM et TP)
%\newcommand{\UPSTIduree}{3h}

% Note de bas de première page
%\newcommand{\UPSTInoteBasDePremierePage}{Note de bas de 1ère page}

% Numéro (ajoute " n°1" après DS ou DM)
%\newcommand{\UPSTInumero}{1}

% Numéro chapitre
%\newcommand{\UPSTInumeroChapitre}{4}

% En-tête customisé
%\newcommand{\UPSTIenTetePrincipalCustom}{UPSTIenTetePrincipalCustom}

% Message sous le titre
%\newcommand{\UPSTImessage}{Message sous le titre}

% Référence au programme
%\newcommand{\UPSTIprogramme}{\EPBComp \EPBCompP{B1-02}, \EPBCompP{B2-49}, \EPBCompS{B2-50}, \EPBCompS{B2-51}, \EPBCompP{C1-07}, \EPBCompP{C1-08}}

% Si l'auteur n'est pas l'auteur par défaut
\renewcommand{\UPSTIauteur}{Equipe pédagogique}

% Si le document est réalisé au nom de l'équipe
%\newcommand{\UPSTIdocumentCollegial}{1}

% Source
%\newcommand{\UPSTIsource}{UPSTI}

% Version du document
\newcommand{\UPSTInumeroVersion}{1}

%-----------------------------------------------
\UPSTIcompileVars		% "Compile" les variables
%%%%%%%%%%%%%%%%%%%%%%%%%%%%%%%%%%%%%%%%%%%%%%%%
% Avant d'entrer en classe
% Portable rangé
% Couvre chefs
% Oreillette
%
% Casque moto non visible
% Sac au sol
% Enlever son manteau
%
% En entrant :
% Bonjour
%
% On ne boit pas ni ne mange en classe
%%%%%%%%%%%%%%%%%%%%%%%%%%%%%%%%%%%%%%%%%%%%%%%%
% Début du document
%%%%%%%%%%%%%%%%%%%%%%%%%%%%%%%%%%%%%%%%%%%%%%%%
\begin{document}
\UPSTIbuildPage
\title{}
\date{Année 2018 - 2019}

Suite à l'agression de deux enseignantes vendredi 7 février, le personnel du lycée a jugé indispensable de suspendre les cours le lundi 11 février, la sécurité des élèves et des enseignants n'étant plus assurée.

Une enquête policière est en cours, il y a eu dépôt de plainte.

Cette journée de concertation a permis d'analyser de nombreux dysfonctionnements au sein de l'établissement. L'intégralité de la communauté éducative a constaté un manquement régulier des élèves aux règles de vie élémentaires. Celles-ci doivent être appliquées par tous les élèves pour le bon fonctionnement de l'établissement et la sécurité de tous.

C'est pourquoi nous jugeons nécessaire de vous rappeler quelques points fondamentaux de la conduite attendue d'un élève. %droits et les devoirs de chacun au sein du lycée.

\section{Avant d'entrer en classe}

\begin{itemize}
  \item J'éteins et je range mon téléphone portable,
  \item J'enlève
  \begin{itemize}
    \item casquette
    \item capuche
    \item bonnet
    \item casque
    \item \dots
  \end{itemize}
  \item Je retire mes écouteurs,
  \item Je suis à l'heure en cours. %devant la porte de la salle avant la deuxième sonnerie\item
\end{itemize}

\section{En entrant en classe}
\begin{itemize}
  \item Je rentre calmement,
  \item Je dis bonjour,
  \item J'enlève mon manteau et mes gants,
  \item Je sors mon matériel de travail et uniquement mon matériel de travail,
  \item Je pose mon sac au sol.
\end{itemize}

\section{Durant toute l'heure de cours}

\begin{itemize}
  \item Je me tiens correctement sur ma chaise,
  \item Je ne mange ni ne bois en classe,
  \item Je ne me lève pas sans autorisation,
  \item Je ne prends la parole qu'avec autorisation,
  \item Je ne sors pas de la salle sans autorisation,
  \item Je range ma chaise,
  \item Je ne dégrade pas le matériel.
\end{itemize}


\section{Avant de sortir de la salle}
\begin{itemize}
  \item Je m'assure de la propreté de la salle,
  \item Une fois le cours terminé, je ne traine pas dans les couloirs,
  \item Je descends dans la cour pendant les récréations,
  \item Je descends dans la cour lorsque je n'ai pas cours.
\end{itemize}
Tout manquement par un élève à l'un des éléments cités dans ce document ou dans le règlement intérieur du lycée l'expose à une punition ou une sanction.

À l'inverse, le respect régulier de l'intégralité de ces points sera valorisé par l'équipe pédagogique.

\end{document}
