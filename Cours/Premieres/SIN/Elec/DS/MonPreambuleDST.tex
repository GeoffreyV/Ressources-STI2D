%% La partie du code technique est due � 100% � Jean-C�me Charpentier.



% Les packages n�cessaires pour faire fonctionner les macros qui suivent
\usepackage{xlop}
\usepackage{ifthen}
\usepackage{calc}
\usepackage{xcolor}
\usepackage{lastpage}
\usepackage{fancyhdr}


%%%%%%%%%%%%%%%%%%%%%%%%%%%%%%%%%%%%%%
% A adapter en fonction de ses go�ts %
%%%%%%%%%%%%%%%%%%%%%%%%%%%%%%%%%%%%%%
\newcommand{\MesConsignes}{%
    \begin{center}%
        \begin{minipage}{0.7\textwidth}%
            \footnotesize\slshape%
             Le soin et la rédaction seront pris en compte dans la
             notation. \\
             \textbf{Pour répondre à chaque question, vous écrirez l'expression littérale avant toute application numérique.}
             Le barème est approximatif.
        \end{minipage}%
    \end{center}%
}



\newcommand{\EnteteDst}[3][6cm]{%
    % #1 = largeur des pointill�s devant "Nom". Argument optionnel, par d�faut #1 = 6cm
    % #2 = classe
    % #3 = date
    \pagestyle{fancy}%
    \lhead{\ifthenelse{\thepage=1}{NOM Prénom : \makebox[#1]{\dotfill} \qquad #2}{#2}}%
    \rhead{#3}%
    \cfoot{\thepage/\pageref{LastPage}}%
}



%%%%%%%%%%%%%%%%%%%%%%%%%%%%%%%%%%%%%%%%%%%%%%%%%%%%%%%%%%%%%%%%%%%%%%%%
% Grand merci � Jean-C�me Charpentier pour la suite... qui me d�passe. %
%%%%%%%%%%%%%%%%%%%%%%%%%%%%%%%%%%%%%%%%%%%%%%%%%%%%%%%%%%%%%%%%%%%%%%%%

\opset{decimalsepsymbol={,}}  % option du package xlop
\newlength{\fulllinewidth}
\AtBeginDocument{\setlength{\fulllinewidth}{\linewidth}}
\newlength{\brouillon}

\newif\ifBaremeDetail
\BaremeDetailfalse

\newcommand*\FranPt[1]{%
   \ifBaremeDetail
     #1 pt%
     \ifdim#1pt<2pt % singulier lorsque < 2 et pas pluriel lorsque > 1
     \else
       s%
     \fi
   \fi
}

\makeatletter

\newcommand*{\taille@Exo}{}
\newcommand*{\tailleEXO}[1]{\renewcommand*{\taille@Exo}{#1}}
\newcommand{\BaremeEspace@Marge}{5pt}
\newcommand{\BaremeEspaceMarge}[1]{\renewcommand{\BaremeEspace@Marge}{#1}}


\newcommand*{\brm}[1]{%
   % (\BaremeEspace@Marge == 0pt) => au ras de la marge gauche
   % (\BaremeEspace@Marge >0pt) => plus � l'int�rieur de la marge gauche
   \immediate\write\@auxout{%
     \string\opadd*{#1}%
                   {total\the\c@section-\the\c@exo}%
                   {total\the\c@section-\the\c@exo}%
   }%
   \setlength{\brouillon}{\BaremeEspace@Marge-\linewidth+\fulllinewidth}%
   \makebox[0pt][r]{\color{gray}\FranPt{#1}\hspace*{\brouillon}}%
   \ignorespaces%
}

\newcounter{exo}[section]
% D�commenter pour une num�rotation type section.exo
%\renewcommand\theexo{\thesection.\arabic{exo}}
\newcommand\EXO{%
   \refstepcounter{exo}%
   \immediate\write\@auxout{%
     \string\opcopy{0}{total\the\c@section-\the\c@exo}%
   }%
   \@ifstar\exercice@\exercice@@
}

\newcommand\exercice@@[1][]{%
   \csname \taille@Exo section\endcsname*{Exercice \theexo #1%
   \hfill \total{\the\c@section-\the\c@exo}}
}

\newcommand\exercice@[1][]{%
   \csname \taille@Exo section\endcsname*{Exercice \theexo #1}

}

\newcommand\total[1]{%
   \@ifundefined{Op@total#1}{recompilez}%
   {%
     \opunzero{total#1}% on ne sait jamais
     \opprint{total#1}~%
     \opcmp{total#1}{2}%
     \ifopge points\else point\fi
   }%
}

\makeatother
