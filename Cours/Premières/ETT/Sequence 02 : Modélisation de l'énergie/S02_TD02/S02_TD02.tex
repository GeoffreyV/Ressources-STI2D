\documentclass[10pt,fleqn]{article} % Default font size and left-justified equations
\usepackage[%
    pdftitle={SLCI : Transformée de Laplace},
    pdfauthor={Xavier Pessoles}]{hyperref}

\input{Cours/style/new_style}
\input{Cours/style/macros_SII}
%% Ça, ce sont les commandes publiques %%
\newcommand{\RemplirTrou}{false}
\newcommand{\TailleTrou}{\Large}
\newcommand{\ColorTrou}{black}
\newlength{\DeltaTrou}
\setlength{\DeltaTrou}{0pt}
%%%%%%%%%%%%%%%%%%%%%%%%%%%%%%%%%%%%%%%%%

\newsavebox{\totoTrou}
\newsavebox{\tataTrou}
\newlength{\longpointTrou}
\newlength{\profondeurTrou}
\newlength{\hauteurTrou}


\newcommand{\trou}[2][\RemplirTrou]{%
    \savebox{\tataTrou}{\hbox{\color{\ColorTrou}.}}%
    \leavevmode%
    \ifthenelse{\equal{#1}{false}}{%
        \ifmmode%
            \ifinner%
                \savebox{\totoTrou}{\TailleTrou$#2$}%
                \settodepth{\profondeurTrou}{\TailleTrou$#2$}%
                \settoheight{\hauteurTrou}{\TailleTrou$#2$}%
            \else%
                \savebox{\totoTrou}{\TailleTrou$\displaystyle#2$}%
                \settodepth{\profondeurTrou}{\TailleTrou$\displaystyle#2$}%
                \settoheight{\hauteurTrou}{\TailleTrou$\displaystyle#2$}%
            \fi%
        \else%
            \savebox{\totoTrou}{\TailleTrou#2}%
                \settodepth{\profondeurTrou}{\TailleTrou#2}%
                \settoheight{\hauteurTrou}{\TailleTrou#2}%
        \fi%
        \setlength{\longpointTrou}{0pt}%
        \whiledo{\longpointTrou<\wd\totoTrou}{%
            \hbox{%\TailleTrou%\rule{0pt}{\baselineskip}% <-- version d'origine
                  \rule[-\profondeurTrou-\DeltaTrou]{0pt}{\profondeurTrou+\hauteurTrou+2\DeltaTrou}% <-- ma version
                  \usebox{\tataTrou}%
            }% fin de \hbox
            \penalty0\advance\longpointTrou by\wd\tataTrou%
        }% fin du whiledo
    }% fin du then de ifthenelse
    {%
        {\textbf{\color{red}#2}}%
    }% fin du else de ifthenelse
}% fin de la définition de \trou


\usepackage{siunitx}
%\fichetrue
\fichefalse

%\proftrue
%\proffalse

\tdtrue
%\tdfalse

%\courstrue
\coursfalse

% -------------------------------------
% Déclaration des titres
% -------------------------------------

\def\discipline{Enseignement \\Technologique \\ Transversal}
\def\xxtete{Enseignement Technologique Transversal}

\def\classe{1 STI2D}
\def\xxnumpartie{Séquence 2}
\def\xxpartie{Comparatif énergie pétrole et solaire}

\def\xxnumchapitre{Séance 2}
\def\xxchapitre{\hspace{.12cm} Énergies, Puissances et rendement}

\def\xxposongletx{2}
\def\xxposonglettext{1.45}
\def\xxposonglety{23}
\def\xxonglet{Seq. 2 -- Se. 2}

\def\xxactivite{TD}
\def\xxauteur{\textsl{Geoffrey Vaquette}}

\def\xxcompetences{%
\textsl{%
\textbf{Savoirs et compétences :}
\begin{itemize}[label=\ding{112},font=\color{ocre}]
\item CO2.1	Identifier les flux et la forme de l'énergie, caractériser ses transformations et/ou modulations et estimer l'efficacité globale d'un système.
\end{itemize}
%
}}

\def\xxfigures{
\begin{center}
\includegraphics[height=4cm]{images/smartphone.png} \\
\includegraphics[height=4cm]{images/petrole.png} \\
\includegraphics[height=4cm]{images/gaz.png} \\
\end{center}
}%figues de la page de garde
\def\xxpied{%
Energies, Puissances et Rendement \xxactivite%
}

%---------------------------------------------------------------------------

\renewcommand{\RemplirTrou}{true}
\begin{document}
\chapterimage{png/Fond_solaire}
\input{Cours/style/new_pagegarde}

\begin{obj}
  Analyser et comparer les apports d’énergies de différentes sources
A travers cette activité, vous serez amené à comparer une source
d’énergie renouvelable et non renouvelable afin d’en faire ressortir les
avantages et inconvénients de chacune.
\end{obj}
\section{Un constat}
 Environ 50 \% des réserves énergétiques mondiales de pétrole ont déjà
été pompées. Les réserves restantes connues sont estimées à \SI{164.4}{
milliards de tonnes}

\section{Comparaison entre l’énergie solaire et le pétrole}

\begin{exercise}
\begin{enumerate}
\item Sachant que la masse volumique moyenne du pétrole est
$\mu_p = \SI{0,8275}{kg/L}$, calculer le volume $V_p$ en litre de pétrole encore
disponible.
\begin{corrige}
On a une masse $m_p = \SI{164.4 10^{3}}{MT}$ de pétrole. Connaissant la masse volumique du pétrole, on peut déduire : $$ m_p = \mu_p \times V_p \RightArrow V_p = \frac{m_p}{\mu_p} = \frac{164.4 10^3 10^6 10^3}{0.8275} = 1.98 10^14 $$
\end{corrige}
\item Sachant que 1 baril de pétrole correspond à 159 litres (1bl = \SI{159}{L}),
calculer le nombre de barils de pétrole $N_p$ encore disponibles.
\begin{corrige}
$$N_p = $$
\end{corrige}
\item Sachant que la consommation actuelle de pétrole est d’environ
80 Mbl/jour, calculer en combien de jour les réserves seront
épuisées. En déduire dans combien d’année il n’y aura plus de
pétrole au rythme actuel.
Une des sources d'énergie extérieures à la Terre et utilisable
actuellement est le soleil. Le soleil rayonne par an une énergie de
16.1015 kWh.
\item Combien faudrait-il de baril de pétrole pour avoir une énergie
équivalente. Comparer aux réserves restantes.
\item Conclure
\end{enumerate}
\end{exercise}



\end{document}
